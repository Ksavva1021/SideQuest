\chapter{Conclusions}
\chaptermark{Conclusions}  
\thispagestyle{plain}  % First page has default style
\pagestyle{chapterpages}
\label{Section:Conclusions}

This thesis has presented work ranging from computing strategies to Higgs boson physics to extended Higgs sector physics, all carried out within the \ac{CMS} experiment at the \ac{LHC}. The studies addressed three main aspects. First, the development and validation of an alternative event data format, designed to improve the long-term sustainability of the CMS computing model and enable the experiment to meet its ever-growing computing demands. Second, a search for extended Higgs sector signatures in final states with multiple tau leptons, providing direct constraints on new physics scenarios beyond the \ac{SM}. Third, a precision measurement of the CP structure of the Higgs–tau Yukawa coupling. Taken together, these analyses illustrate both the technical and physics challenges faced by CMS, as well as the innovative strategies developed to address them.

The study of the RAW$'$ data format demonstrated the potential of a more compact representation of \ac{CMS} collision events. While the first iteration suffered from serious drawbacks, particularly in displaced track reconstruction, refinements introduced in the second version successfully recovered much of the lost performance. With these improvements, RAW$'$ has emerged as a compelling candidate format for future use, especially in the \ac{HL}-\ac{LHC} era. Its deployment in heavy-ion collisions since 2024 underscores its practicality, and its adoption for pp data-taking is expected in the near future.

The search for extended Higgs sectors in the four-tau final state was performed with the full Run 2 CMS dataset, corresponding to $137\unit{fb}^{-1}$ of pp collisions at $\sqrt{s} = 13$\TeV. No significant excess beyond the \ac{SM} background expectation was observed. The analysis set upper limits on the production cross section times branching fraction for the process $Z^* \to \phi A \to \PGt^+\PGt^-\PGt^+\PGt^-$, ranging from $\mathcal{O}(190\unit{fb})$ at low masses to $\mathcal{O}(0.4\unit{fb})$ at high $(m_A, m_\phi)$. These results exclude a sizeable region of parameter space in the Type-X \ac{2HDM}, including scenarios motivated by the muon anomalous magnetic moment. A dedicated journal article is expected to be submitted to \textit{PRL}.

The measurement of the CP structure of the Higgs–tau Yukawa coupling was performed in the $\tauh\tauh$ final state with Run 3 data, exploiting improved tau identification, triggering, event categorisation, and refined acoplanarity reconstruction methods. The analysis excluded the pure CP-odd hypothesis at $1.74\sigma$ significance, compared to an expectation of $2.39\sigma$, with the observed sensitivity limited by statistical fluctuations. A direct comparison with the Run 2 result demonstrated clear improvements in per-category sensitivity despite reduced statistical power, reflecting the success of methodological refinements. Finally, the statistical combination of Run 2 and Run 3 results provided the most precise determination to date of the mixing angle $\alpha^{\PH\tau\tau}$, disfavoring the pure CP-odd scenario at $3.55\sigma$ and yielding an uncertainty of $\pm 15^\circ$. This combined measurement represents a significant step forward in constraining the CP properties of the Higgs sector, while remaining consistent with the \ac{SM} prediction of a CP-even coupling. A dedicated paper on this analysis is planned once the complementary $\tau_e\tau_h$ and $\tau_\mu\tau_h$ channels are finalised, with a full publication incorporating the complete Run 3 dataset foreseen after data-taking concludes in 2026.