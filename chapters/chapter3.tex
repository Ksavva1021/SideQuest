\chapter{LHC and the CMS experiment}
\chaptermark{The LHC and the CMS experiment}  
\thispagestyle{plain}  % First page has default style
\pagestyle{chapterpages}
\label{Section:Chapter3}

The physics analyses presented in this thesis are performed using data generated by the LHC and collected by the \ac{CMS} experiment. This chapter begins by exploring how the LHC accelerates and collides protons up to centre-of-mass energies ($\sqrt{s}$) of $13.6\TeV$, creating the extreme but necessary conditions needed for rigorous tests of the SM at the electroweak scale. The discussion then shifts to the CMS detector, a multipurpose apparatus composed of several layers of specialised subdetectors. These intricate systems work in concert to enable the precise reconstruction of particles emerging from proton-proton collisions at the heart of the detector.

\section{The LHC}

The LHC~\cite{LHC_1}, situated at the European Organization for Nuclear Research (CERN) near Geneva, Switzerland, stands as a testament to human scientific achievement. Ingeniously, the LHC was placed within the tunnel previously occupied by the Large Electron Positron (LEP) collider. Engineered to facilitate proton-proton (pp) collisions, the machine was designed to generate such collisions with a centre-of-mass energy up to $14\TeV$ and an unprecedented luminosity of $10^{34}\unit{cm}^{-2}\unit{s}^{-1}$. As mentioned earlier, the LHC is currently running just below its maximum energy capabilities, with the each beam carrying $6.8\TeV$ of energy. Spanning a circumference of $27\unit{km}$, the LHC mirrors the basic layout of its predecessor, LEP, as illustrated in Fig~\ref{Figure:Chapter3_LHC_BasicLayout}. 

\begin{figure}[h]
\centering
\includegraphics[width= 0.7\textwidth]{Figures/Chapter3/LHC_BasicLayout.jpg}
\caption{Basic schematic layout of the LHC consisting of 8 arc sections along with the two circulating beams and the ATLAS, ALICE, CMS, and LHCb experiments~\cite{LHC_BasicLayout}.}
\label{Figure:Chapter3_LHC_BasicLayout}
\end{figure}

The experimental landscape is strategically arranged with two general purpose experiments, ATLAS~\cite{LHC_ATLAS} and CMS~\cite{LHC_CMS}, positioned at diametrically opposite sections, at Points 1 and 5 respectively. The ALICE experiment~\cite{LHC_ALICE} occupies Point 2, while LHCb~\cite{LHC_LCHb} is situated at Point 8. At each of these critical locations, the circulating beams are precisely focused and brought into collision.

Particles are not accelerated to an energy of $6.8~\TeV$ per beam simply by circulating in the LHC tunnel. Instead, the LHC is the final stage of a sophisticated accelerator chain~\cite{LHC_InjectorComplex} at CERN, in which a series of machines successively boost the energy of the particles, as shown in Fig.~\ref{Figure:Chapter3_LHC_Complex}. The first step in this accelerator complex is the Linear Accelerator 4 (LINAC4)~\cite{LINAC4}, which produces a beam of negative hydrogen ions ($\text{H}^-$), each consisting of a proton and two electrons. LINAC4 accelerates these ions to an energy of $160\MeV$ before injecting them into the Proton Synchrotron Booster (PSB). During injection, the electrons are stripped off, leaving behind bare protons. The PSB then accelerates the protons to a kinetic energy of $2.0\GeV$. Next, the beam is transferred to the Proton Synchrotron (PS), which increases their energy to $26\GeV$. From there, the particles are sent to the largest machine in the injector complex, the Super Proton Synchrotron (SPS), which boosts their energy to $450\GeV$. Throughout the injector chain, the protons are grouped into bunches, which are injected into the two concentric beam-pipes of the LHC as two counter-rotating beams. The injection process continues until each beam consists of 2808 bunches, separated by $25\unit{ns}$.

\begin{figure}[h]
\centering
\includegraphics[width= 1.0\textwidth]{Figures/Chapter3/LHC_AcceleratorComplex.png}
\caption{Schematic diagram of the CERN accelerator complex~\cite{LHC_InjectorComplex}.}
\label{Figure:Chapter3_LHC_Complex}
\end{figure}

A network of 1232 superconducting (niobium-titanium) dipole magnets (niobium-titanium) bend the beams along the circular LHC ring. These magnets operate at ultra-low temperatures of $1.9\unit{K}$, achieved using superfluid helium, and are able of producing magnetic fields up to $8.4\unit{T}$. Beam acceleration is facilitated by sixteen $400\unit{MHz}$ radiofrequency (RF) cavities, providing a \~14-fold energy increase in comparison to the injection energy. The circular trajectory of the accelerating beams is maintained by dynamically adjusting the magnetic field strength of the dipole magnets while 392 quadrupole are utilised to focus the beams. Just before collision at each of the four principal points, four inner triplet quadrupole magnets are used to reduce the transverse size of the beams. This effectively squeezes them, aiming to maximise the collision rate~\cite{LHC_Run3}, as illustrated in Fig.~\ref{Figure:Chapter3_LHC_BeamSqueeze}. These values reflect the state of the accelerator complex following the upgrades carried out during the latest scheduled long shutdown, which was completed in 2021.

\begin{figure}[h]
\centering
\includegraphics[width= 0.7\textwidth]{Figures/Chapter3/LHC_BeamSqueeze.pdf}
\caption{Illustration of beam squeezing prior to colliding in the heart of detectors at the LHC.}
\label{Figure:Chapter3_LHC_BeamSqueeze}
\end{figure}

\subsubsection{Cross section and Luminosity}

Cross section and luminosity are the key quantities essential in understanding and measuring what happens in collider collisions. Together, these determine how often specific physical processes occur in the detector.

The cross section, $\sigma$, describes the probability of a particular process of interesting taking place during a collision. As the beams cross in the heart of the LHC detectors, many different processes can occur. The cross section of each of these processes depends on the type and the energy of the colliding particles, $\sigma(\sqrt{s})$. To calculate how often a process of interest occurs at a collider, the cross section is combined with the instantaneous luminosity, $\mathscr{L}$. Instantaneous luminosity is measure of how tightly particles are packed into a given space and it is a function of the beam parameters~\cite{LHC_HL},


\begin{equation}
\begin{aligned}
    \mathscr{L} &= \gamma \frac{n_b N^2 f_{\text{rev}}}{4\pi \beta^* \epsilon_n} R \\
    R &= 1 / \sqrt{1 + \left( \frac{\theta_c \sigma_z}{2\sigma} \right) }
\end{aligned}
\end{equation}

where $\gamma$ is the proton beam energy expressed in rest mass units, $n_b$ is the number of bunches per beam, $N$ is the number of particles per bunch, $f_{rev}$ is the revolution frequency ($11.2\unit{kHz}$), $\beta^*$ is the beam beta function at the collision point and $\epsilon_n$ is the normalised tranverse beam emittance. The term R is a geometrical reduction factor for luminosity. This is expressed as a function of the crossing angle of the beams at the collision point ($\theta_c$) and the transverse (longitudinal) spread of the particle bunch, $\sigma_z(\sigma)$. 

The cross section, together with the instantaneous luminosity, determines the rate, R, at which a given process occurs in the detector,

\begin{equation}
    R = \sigma(\sqrt{s}) \cdot \mathscr{L} 
\end{equation}

The beam parameters are constant over time hence, the luminosity also varies over time. Therefore, a more appropriate quantity is the integrate luminosity, $\mathscr{L}_{int}$, which considers the total data collected over a given a period.

\begin{equation}
    \mathscr{L}_{int} = \int \mathscr{L}(t) dt
\end{equation}

In a given period, the total number of events, N, for a given process with cross section $\sigma$ can be expressed as:

\begin{equation}
    \mathscr{N} = \sigma \cdot \mathscr{L}_{int}
\end{equation}

A summary of the total integrated luminosity delivered to the CMS experiment is provided in Fig.~\ref{Figure:Chapter3_CMS_IntegratedLumi}.

\begin{figure}[h]
\centering
\includegraphics[width= 0.7\textwidth]{Figures/Chapter3/CMS_IntegratedLumi.pdf}
\caption{Total integrated luminosity delivered to the CMS experiment between 2015 to 2024. Data collected from all periods except 2015 and 2024 are used in this thesis. Figure is taken from Ref.~\cite{CMS_IntegratedLumi}.}
\label{Figure:Chapter3_CMS_IntegratedLumi}
\end{figure}

\subsubsection{Pileup}

The CMS experiment is set out to investigate the rarest interactions of proton collisions. In the effort of maximising the chances of collisions, the LHC collides large bunches of protons rather than single protons, as discussed earlier. However, this also comes with problems because multiple protons are interacting when the bunches collide. In other words, a detector such as CMS records particles from the interaction of interest along with particles original from multiple additional interactions, called pileup interactions. Increasing the instantaneous luminosity leads to the enhancement of pileup. The removal of these unwanted overlap collisions requires increasingly more sophisticated techniques. Figure~\ref{Figure:Chapter3_CMS_Pileup} summarises the pileup conditions between 2015 to 2024, as seen by the CMS experiment.

\begin{figure}[h]
\centering
\includegraphics[width= 0.7\textwidth]{Figures/Chapter3/CMS_Pileup.pdf}
\caption{Distribution of average number of interactions per crossing for pp collisions between 2015 to 2024 using data from the CMS experiment. Figure is taken from Ref.~\cite{CMS_IntegratedLumi}.}
\label{Figure:Chapter3_CMS_Pileup}
\end{figure}


